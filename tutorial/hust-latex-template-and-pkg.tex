
\documentclass[langauge=chinese]{hustbeamer}

\usepackage{metalogo,tabu}
\usepackage{tabularx}
\usepackage{minted}
\usepackage{framed}

\definecolor{minted_bg}{rgb}{0.95,0.95,0.95}
\newminted{latex}{bgcolor=minted_bg,frame=lines,fontsize=\footnotesize}
\newminted{bat}{bgcolor=minted_bg,frame=lines,fontsize=\scriptsize}
\newminted{make}{bgcolor=minted_bg,frame=lines,fontsize=\tiny}

\graphicspath{ {.}{./fig/} }

\title{\LaTeX{}模板和常用宏包介绍}
\author{许铖}
\institute{\email{xucheng@hust.edu.cn}}
\date{2013}{11}{21}

\begin{document}

\maketitle
\PrintTOC

\section{模板介绍}

\begin{frame}[fragile]{\secname}{模板地址}
这些是我制作的模板:

\noindent\url{https://github.com/xu-cheng/hust-latex-template}

\begin{description}
\footnotesize
    \item[hustthesis] \url{https://xu-cheng.github.io/hustthesis}
    \item[hustreport] \url{https://xu-cheng.github.io/hustreport}
    \item[hustbeamer] \url{https://xu-cheng.github.io/hustbeamer}
\end{description}

\begin{block}{提醒}
使用前请阅读模板文档及示例。
\end{block}
\end{frame}

\begin{frame}[fragile]{\secname}{模板安装}

{\Large 使用模板环境要求:}
\begin{description}
    \item[\LaTeX{}环境] 安装最新版本的 TeXLive(推荐)或 MiKTeX,\newline\textbf{不要}使用CTeX。确保所有宏包都更新至最新。
    \item[字体] 安装如下中文字体:\\
    \begin{itemize}
        \item AdobeSongStd-Light
        \item AdobeKaitiStd-Regular
        \item AdobeHeitiStd-Regular
        \item AdobeFangsongStd-Regular
    \end{itemize}
\end{description}

\end{frame}

\begin{frame}[fragile]{\secname}{模板安装}
安装模板使用如下命令:
\mint[bgcolor=minted_bg]{bash} |make install|
使用如下命令卸载:
\mint[bgcolor=minted_bg]{bash} |make uninstall|

对于没有安装Make的Windows用户,安装命令对于如下:
\mint[bgcolor=minted_bg]{bat} |makewin32.bat install|
卸载命令如下:
\mint[bgcolor=minted_bg]{bat} |makewin32.bat uninstall|
\end{frame}

\begin{frame}[fragile]{\secname}{编译命令}

只能使用\LuaLaTeX{}(推荐)和\XeLaTeX{}编译该模板的文档。\pause

不能使用\LaTeX{}或pdf\LaTeX{}编译。\pause

更不能使用\TeX{},pdf\TeX{},\LuaTeX{}或\XeTeX{}编译。\pause

\begin{center}
\begin{tabular}{cc}
引擎 & 扩展 \\ \hline
\textrm{\TeX} & \textrm{\LaTeX} \\
\textrm{pdf\TeX} & \textrm{pdf\LaTeX} \\
\textrm{\XeTeX} & \textrm{\XeLaTeX} \\
\textrm{\LuaTeX} & \textrm{\LuaLaTeX} 
\end{tabular}
\end{center}
\end{frame}

\begin{frame}[fragile]{\secname}{模板使用}

在源文件开头处选择加载文档类型,即可使用本模板,如:

\begin{latexcode}
\documentclass[options..]{hustthesis}
\documentclass[options..]{hustreport}
\documentclass[options..]{hustbeamer}
\end{latexcode}

\end{frame}

\begin{frame}[fragile]{\secname}{模板使用}

加载模板时,可用选项如下:\\[1.2ex]

\begin{tabularx}{.8\textwidth}{|l|X|}
\hline
\multicolumn{1}{|c|}{\textbf{模板}} & \multicolumn{1}{c|}{\textbf{可用选项}} \\ \hline
hustthesis & langauge,format,degree \\ \hline
hustreport & langauge,format,category \\ \hline
hustbeamer & langauge \\ \hline
\end{tabularx}

\end{frame}

\begin{frame}[fragile]{\secname}{模板使用示例}
\begin{latexcode*}{fontsize=\tiny}
\documentclass[degree=phd,language=chinese]{hustthesis}

\stuno{你的学号}
\schoolcode{10487}
\title{中文标题}{英文标题}
\author{作者名}{作者名拼音}
\major{专业中文}{专业英文}
\supervisor{指导老师中文}{指导老师英文}
\date{2013}{7}{1} % 答辩日期

\zhabstract{中文摘要}
\zhkeywords{中文关键字}
\enabstract{英文摘要}
\enkeywords{英文关键字}
\end{latexcode*}
\end{frame}

\begin{frame}[fragile]{\secname}{模板使用示例}
\begin{latexcode*}{fontsize=\tiny}
\begin{document}
\frontmatter
\maketitle
\makeabstract
\tableofcontents
\listoffigures
\listoftables

\mainmatter
%% 正文

\backmatter
\begin{ack}
%% 致谢
\end{ack}
\bibliography{参考文献.bib文件}
\appendix
\begin{publications}
%% 发表过的论文列表
\end{publications}
%% 附录剩余部分
\end{document}
\end{latexcode*}
\end{frame}

\section{常用宏包介绍}

\begin{frame}{\secname}{\textrm{\LaTeX}资料}
\begin{enumerate}
    \item 《一份不太简短的\textrm{\LaTeXe}介绍》\mint{bash}|texdoc lshort-zh|
    \item \url{http://en.wikibooks.org/wiki/LaTeX/}
    \item \url{http://tex.stackexchange.com/} 和 Google
\end{enumerate}
\pause
\begin{block}{texdoc命令}
    texdoc命令用于查看\textrm{\LaTeX}文档,使用方法如下:\\*
    \hspace{.8em}\texttt{texdoc\hspace{.8em}\emph{<package name>}}
\end{block}
\end{frame}

\begin{frame}[fragile]{\secname}{中文字体 --- luatexja}
\begin{description}
    \item[宏包] luatexja
    \item[文档] \mint{bash}|texdoc luatexja-zh| 
    \item[注释] \LuaTeX{}下的宏包。如果你使用\XeLaTeX{},有类似宏包xeCJK。\pause
    \item[示例] \hfill\\
\begin{latexcode*}{fontsize=\tiny}
\usepackage{luatexja-fontspec}
% 英文字体
\setmainfont[Ligatures={Common,TeX}]{Tex Gyre Termes}
\setsansfont[Ligatures={Common,TeX}]{Droid Sans}
\setmonofont{CMU Typewriter Text}
\defaultfontfeatures{Mapping=tex-text,Scale=MatchLowercase}
% 中文字体
\setmainjfont[BoldFont={AdobeHeitiStd-Regular},
              ItalicFont={AdobeKaitiStd-Regular}]
             {AdobeSongStd-Light}
\setsansjfont{AdobeKaitiStd-Regular}
\defaultjfontfeatures{JFM=kaiming}
\end{latexcode*}
\end{description}
\end{frame}

\begin{frame}[fragile]{\secname}{算法环境 --- algorithm2e}
\begin{description}
    \item[宏包] algorithm2e
    \item[文档] \mint{bash}|texdoc algorithm2e| \pause
    \item[示例] \hfill\\
\begin{latexcode*}{fontsize=\tiny}
\begin{algorithm}[H]
\SetAlgoLined
\KwData{this text}
\KwResult{how to write algorithm with \LaTeX2e }
initialization\;
\While{not at end of this document}{
read current\;
\eIf{understand}{
go to next section\;
current section becomes this one\;
}{
go back to the beginning of current section\;
}
}
\caption{How to write algorithms}
\end{algorithm}
\end{latexcode*}
\end{description}
\end{frame}

\begin{frame}[fragile]{\secname}{算法环境 --- algorithm2e}
\scalebox{0.65}{
\begin{algorithm}[H]
\SetAlgoLined
\KwData{this text}
\KwResult{how to write algorithm with \LaTeX2e }
initialization\;
\While{not at end of this document}{
read current\;
\eIf{understand}{
go to next section\;
current section becomes this one\;
}{
go back to the beginning of current section\;
}
}
\caption{How to write algorithms}
\end{algorithm}    
}
\end{frame}

\begin{frame}[fragile]{\secname}{定理证明环境 --- ntheorem}
\begin{description}
    \item[宏包] ntheorem
    \item[文档] \mint{bash}|texdoc ntheorem|
    \item[注释] 使用时,需仔细阅读文档,避免宏包冲突。 \pause
    \item[示例] \hfill\\
\begin{latexcode*}{fontsize=\tiny}
\usepackage[amsmath,amsthm,thmmarks,hyperref,thref]{ntheorem}
\theoremstyle{definition}
\newtheorem{definition}{定义}[chapter]
\theoremstyle{plain}
\newtheorem{theorem}{定理}[chapter]

\begin{definition}
This is a definition.
\end{definition}
\begin{theorem}
This is a theorem.
\end{theorem}
\end{latexcode*}
\end{description}
\end{frame}

\begin{frame}[fragile]{\secname}{定理证明环境 --- ntheorem}
\begin{definition}
This is a definition.
\end{definition}
\begin{theorem}
This is a theorem.
\end{theorem}
\end{frame}

\begin{frame}[fragile]{\secname}{代码高亮 --- minted}
\begin{description}
    \item[宏包] minted
    \item[文档] \mint{bash}|texdoc minted| 
    \item[注释] 需安装Python和Pygments;编译文档时需要加上参数\verb+-shell-escape+。如:
    \mint{bash}|lualatex -shell-escape input_file|\pause
    \item[示例] \hfill\\
\begin{latexcode*}{fontsize=\tiny}
\begin{minted}[mathescape,linenos,frame=lines]{csharp}
  string title = "This is a Unicode π in the sky"
  /*
  Defined as $\pi=\lim_{n\to\infty}\frac{P_n}{d}$ where $P$ is the perimeter
  of an $n$-sided regular polygon circumscribing a
  circle of diameter $d$.
  */
  const double pi = 3.1415926535
\end{minted}
\end{latexcode*}
\end{description}
\end{frame}

\begin{frame}[fragile]{\secname}{代码高亮 --- minted}
\begin{minted}[mathescape,linenos,frame=lines]{csharp}
  string title = "This is a Unicode π in the sky"
  /*
  Defined as $\pi=\lim_{n\to\infty}\frac{P_n}{d}$ where $P$ is the perimeter
  of an $n$-sided regular polygon circumscribing a
  circle of diameter $d$.
  */
  const double pi = 3.1415926535
\end{minted}
\end{frame}

\begin{frame}[fragile]{\secname}{图片相关宏包}
\begin{center}
\begin{tabularx}{\textwidth}{|l|X|X|}
\hline
\multicolumn{1}{|c|}{\textbf{宏包}} & \multicolumn{1}{c|}{\textbf{作用}} & \multicolumn{1}{c|}{\textbf{注释}} \\ \hline
subcaption & 插入多幅并列的图片 & 不要使用subfigure,subfig这样被废弃的宏包 \\ \hline
overpic  & 在图上层叠其他内容 & \\ \hline
xypic     & 绘制简单流程图 & \\ \hline
tikz/pgf  & 更高级的绘制图片宏包 & 学习难度较大,但绘图质量高 \\ \hline
\end{tabularx}
\end{center}

\pause
\begin{block}{提醒}
插入图片时,请尽量使用矢量图(pdf,eps等格式)。Matlab,matplotlib(Python),R,visio等绘图工具都有相应导出设置。
\end{block}
\end{frame}

\begin{frame}[fragile]{\secname}{表格相关宏包}
\begin{center}
\begin{tabularx}{\textwidth}{|l|X|X|}
\hline
\multicolumn{1}{|c|}{\textbf{宏包}} & \multicolumn{1}{c|}{\textbf{作用}} & \multicolumn{1}{c|}{\textbf{注释}} \\ \hline
tabularx & 更方便调整表格列距 &  \multirow{2}{*}{\tabincell{l}{宏包ltxtable\\合并了这两个功能}} \\ \cline{1-2}
longtable & 插入超长(跨页)表格 & \\ \hline
slashbox  & 在表格中插入斜线 & \\ \hline
excel2latex  & 将excel表格转成latex代码 & 这不是宏包,是一个工具\\ \hline
\end{tabularx}
\end{center}
\end{frame}

\begin{frame}[fragile]{\secname}{其他常用宏包}
\begin{center}
\begin{tabularx}{\textwidth}{|l|X|X|}
\hline
\multicolumn{1}{|c|}{\textbf{宏包}} & \multicolumn{1}{c|}{\textbf{作用}} \\ \hline
enumitem & 自定义列表环境的式样 \\ \hline
fancyhdr & 定义页眉页脚 \\ \hline
fancynum & 将大数每三位断开 \\ \hline
natbib & 定义参考文献格式 \\ \hline
zhnumber & 生成中文数字 \\ \hline
\end{tabularx}
\end{center}
\end{frame}

\begin{frame}
\centering
\Huge{谢\hspace{1em}谢!}
\end{frame}

\end{document}
